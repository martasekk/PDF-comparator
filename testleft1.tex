%%%  Ukázkový text a dokumentace stylu pro text závěrečné (bakalářské a
%%%  diplomové) práce na KI PřF UP v Olomouci
%%%  Copyright (C) 2012 Martin Rotter, <rotter.martinos@gmail.com>
%%%  Copyright (C) 2014 Jan Outrata, <jan.outrata@upol.cz>


%%  Pro získání PDF souboru dokumentu je třeba tento zdrojový text v
%%  LaTeXu přeložit (dvakrát) programem pdfLaTeX.

%%  V případě použití programu BibLaTeX pro tvorbu seznamu literatury
%%  je poté ještě třeba spustit program Biber s parametrem jméno
%%  souboru zdrojového textu bez přípony a následně opět (dvakrát)
%%  přeložit zdrojový text programem pdfLaTeX.

%%  Postup získání Postscriptového souboru je popsán v dokumentaci.


%%  Třída dokumentu implementující styl pro závěrečnou práci. Vybrané
%%  nepovinné parametry (ostatní v dokumentaci):

%%  'master' pro sazbu diplomové práce, jinak se sází bakalářská práce

%%  'program=kód' pro Váš studijní program/obor (specializaci), kódy
%%  pro diplomovou práci 'infoi' pro Informatiku (Obecná informatika),
%%  'infui' pro Informatiku (Umělá inteligence), 'ainfpst' pro
%%  Aplikovanou informatiku (Počítačové systémy a technologie), 'uinf'
%%  pro Učitelství informatiky pro střední školy, 'binf' pro
%%  Bioinformatiku, 'inf' pro Informatiku (bez specializací) a 'ainf'
%%  pro Aplikovanou informatiku (bez specializací), jinak je výchozí
%%  ainfvs pro Aplikovanou informatiku (Vývoj software), a pro
%%  bakalářskou práci 'infoi' pro Informatiku (Obecná informatika),
%%  'itp' pro Informační technologie v prezenční formě, 'itk' pro
%%  Informační technologie v kombinované formě, 'infv' pro Informatiku
%%  pro vzdělávání, 'binf' pro Bioinfomatiku, 'inf' pro Informatiku
%%  (bez specializací), 'ainfp' pro Aplikovanou informatiku (bez
%%  specializací) v prezenční formě, 'ainfk' pro Aplikovanou
%%  informatiku (bez specializací) v kombinované formě, jinak je
%%  výchozí infpvs pro Informatiku (Programování a vývoj software)

%%  'printversion' pro sazbu verze pro tisk (nebarevné logo a odkazy,
%%  odkazy s uvedením adresy za odkazem, ne odkazy do rejstříku),
%%  jinak verze pro prohlížeč

%%  'biblatex' pro zapnutí podpory pro sazbu bibliografie pomocí
%%  BibLaTeXu, jinak je výchozí sazba v prostředí thebibliography

%%  'language=jazyk' pro jazyk práce, jazyky english pro anglický,
%%  slovak pro slovenský, jinak je výchozí czech pro český

%%  'font=sans' pro bezpatkový font (Iwona Light), jinak výchozí
%%  patkový (Latin Modern)

\documentclass[
%  bachelor,
  program=inf,
%  printversion,
%  biblatex,
%  language=english,
%  font=sans,
  figures,
  tables,
  glossaries,
  index
]{kidiplom}
\usepackage{subfig}
\usepackage{graphicx}
\usepackage{mathrsfs}
\usepackage{amsmath}
\usepackage[style=numeric, backend=bibtex]{biblatex}

%% Informace pro úvodní strany. V jazyku práce (pokud není v komentáři
%% uvedeno česky) a anglicky. Uveďte všechny, u kterých není v
%% komentáři uvedeno, že jsou volitelné. Při neuvedení se použijí
%% výchozí texty. Text pro jiný než nastavený jazyk práce (nepovinným
%% parametrem language makra \documentclass, výchozí český) se zadává
%% použitím makra s uvedením jazyka jako nepovinného parametru.

%% Název práce, česky a anglicky. Měl by se vysázet na jeden řádek.
\title{Konstrukce fuzzy regulátorů}
\title[english]{Fuzzy controller design}

%% Volitelný podnázev práce, česky a anglicky. Měl by se vysázet na
%% jeden řádek. Výchozí je prázdný.
\subtitle{}
\subtitle[english]{}

%% Jméno autora práce. Makro nemá nepovinný parametr pro uvedení
%% jazyka.
\author{Martin Hrabal}

%% Jméno vedoucího práce (včetně titulů). Makro nemá nepovinný
%% parametr pro uvedení jazyka.
\supervisor{doc. RNDr. Michal Krupka, Ph.D.}

%% Volitelný rok odevzdání práce. Výchozí je aktuální (kalendářní)
%% rok. Makro nemá nepovinný parametr pro uvedení jazyka.
%\yearofsubmit{\the\year}

%% Anotace práce, včetně anglické (obvykle překlad z jazyka
%% práce). Jeden odstavec!
\annotation{Tato bakalářská práce je zaměřena na proces vývoje fuzzy regulátorů a řešením praktických problémů, které mohou při jejich vývoji vznikat. Součástí práce je i stručný popis řídících systémů, matematická definice fuzzy logiky a sestrojení jednoduchého fuzzy regulátoru pro stabilizaci obráceného kyvadla. Fuzzy systém pro stabilizaci obráceného kyvadla je vytvořen v programu MATLAB, za pomocí rozšíření Simulink a Fuzzy logic toolbox.}

\annotation[english]{This bachelor's thesis is focused on the process of designing fuzzy controllers and dealing with practical problems, that may appear in the process. This thesis also contains short description of control systems, mathematical definition of fuzzy logic and design of a simple fuzzy controller for stabilization of an inverted pendulum. Fuzzy system for the inverted pendulum stabilization is created in the program MATLAB, using Simulink and Fuzzy logic toolbox extensions.}

%% Klíčová slova práce, včetně anglických. Oddělená (obvykle) středníkem.
\keywords{Fuzzy regulátor, fuzzy logika, fuzzy systém, řídící systémy}
\keywords[english]{Fuzzy controller, fuzzy logic, fuzzy system, control systems}

%% Volitelná specifikace příloh textu práce, i anglicky. Výchozí je '1
%% CD/DVD'.
%\supplements{jedno kulaté placaté CD/DVD s malou kulatou dírou uprostřed}
%\supplements[english]{one round flat CD/DVD with a small round hole in the middle}

%% Volitelné poděkování. Stručné! Výchozí je prázdné. Makro nemá
%% nepovinný parametr pro uvedení jazyka.
\thanks{Tímto bych chtěl poděkovat doc. RNDr. Michalu Krupkovi, Ph.D za poskytnutí mnoha vhodných rad a odborné vedení mé práce. }

%% Cesta k souboru s bibliografií pro její sazbu pomocí BibLaTeXu
%% (zvolenou nepovinným parametrem biblatex makra
%% \documentclass). Použijte pouze při této sazbě, ne při (výchozí)
%% sazbě v prostředí thebibliography.
\bibliography{citace}

%% Další dodatečné styly (balíky) potřebné pro sazbu vlastního textu
%% práce.
\usepackage{lipsum}
\usepackage{longtable}

\begin{document}
%% Sazba úvodních stran -- titulní, s bibliografickými údaji, s
%% anotací a klíčovými slovy, s poděkováním a prohlášením, s obsahem a
%% se seznamy obrázků, tabulek, vět a zdrojových kódů (pokud jejich
%% sazba není vypnutá).
\maketitle

%% Vlastní text závěrečné práce. Pro povinné závěry, před přílohami,
%% použijte prostředí kiconclusions. Povinná je i příloha s obsahem
%% přiloženého datového média.

%% -------------------------------------------------------------------

\newcommand{\BibLaTeX}{\textsc{Bib}\LaTeX}


\section{Úvod}
Téma fuzzy logiky jsem si zvolil z části kvůli mému zájmu o robotiku a řídící systémy. Téma mi přišlo zajímavé a myslím si, že se dá uplatnit ve spoustě odvětví průmyslu. Zadáním bakalářské práce bylo popsat kroky pro navrhnutí kvalitního fuzzy regulátoru. V práci jsem se věnoval i obecnému popisu řídících systémů a PID regulátorům, pro které je fuzzy regulátor alternativou. \\ \\
V práci jsem popsal funkci Mamdaniho fuzzy regulátoru a kroky jeho procesu. Poskytnul jsem obecné kroky a rady, kterých bychom se měli při návrhu držet. Také jsem popsal časté problémy, které můžou při návrhu nastat a poskytnul řešení.
Jako součást této práce jsem navrhnul jednoduchý fuzzy regulátor pro řízení obráceného kyvadla a popsal kroky, podle kterých jsem ho navrhnul.

\subsection{Lineární a nelineární systémy}
Systémy můžeme taktéž dělit na lineární, nebo nelineární. Lineární systém je takový, jehož signám můžeme modelovat pomocí lineární funkce \cite{bLCSA}. Uvažujme nějaký systém, který má vstupy $x_{1}(t)$ a $x_{2}(t)$ a výstupy $y_{1}(t)$ a $y_{2}(t)$. U systémů uvažujeme i čas $t$, protože samozřejmě s průběhem času se signál může měnit. Dále předpokládejme, že pro tento systém platí $x_{1}(t) \rightarrow y_{1}(t)$ a $x_{2}(t) \rightarrow y_{2}(t)$, tedy že pro vstup $x_{1}(t)$ nám systém vrátí nějaký výstup $y_{1}(t)$ pro nějakou konstantu času $t$, a podobně pro $x_{2}(t)$ a $y_{2}(t)$. Aby byl systém lineární, musí splňovat následující dvě podmínky \cite{bMCS}\cite{bLCSA}: \paragraph{Additivita} -- výstup pro násobek vstupu musí být stejný jako násobek výstupu pro stejný vstup. Tuto skutečnost můžeme zapsat takto: 
\[ a \cdot x_{1}(t) \rightarrow a \cdot y_{1}(t)\]
\[ a \cdot x_{2}(t) \rightarrow a \cdot y_{2}(t)\]
\paragraph{Homogenita} -- výstup součtu dvou vstupů bude stejný jako součet výstupů pro tyto vstupy jednotlivě:
\[ x_{1}(t) + x_{2}(t) \rightarrow y_{1}(t) + y_{2}(t) \]
Tyto dvě podmínky můžeme zapsat jako jednu:
\[ a \cdot x_{1}(t) + b \cdot x_{2}(t) \rightarrow a \cdot y_{1}(t) + b \cdot y_{2}(t) \]
Nelineární systémy jsou všechny systémy, co tyto podmínky nesplňují. Tato oblast je pro nás zajímavější, protože většina systémů v praxi není lineární. Je ale obtížné takové systémy modelovat pomocí běžných postupů. V praxi se většinou snažíme takový systém přibližně popsat nějakou lineární funkcí, a operovat v nějakém rozumném rozmezí, kde se funkce blíží naší aproximaci. Tento postup užíváme převážně u tzv. PID regulátorů (ovladačů), které jsou v průmyslu dnes velmi rozšířené. Jeden z dalších možných postupů je využít tzv. fuzzy regulátor, který dokáže snadno modelovat nelineární funkce. Oba typy regulátorů si následně popíšeme. 

\subsection{PID regulátor}
Zkratka PID znamená proporcionální, integrační a derivační, což je jméno složek, ze kterých se regulátor skládá. PID regulátor využívá zpětné vazby -- porovnává svou aktuální výstupní hodnotu s požadovanou hodnotou a snaží se co nejpřesněji přiblížit \cite{bPID1}. \\ \\
PID regulátor se využívá v systému s uzavřenou smyčkou -- například pomocí senzoru systém zjistí, v jakém stavu se nachází. Pomocí této informace zjistí chybu a PID regulátor upraví své chování tak, aby se systém choval uspokojivě. Uvažujme například pojízdného robota řízeného PID regulátorem, který má za úkol dojet na nějakou pozici. Je vybaven dvouma pásy.
\begin{figure}[h]
\centering
\includegraphics[scale=1]{images/robot.png}
\caption{Robot}
\end{figure}

\[ y_{P}(t) = k \cdot (y_{s}(t)-y(t)) \] } 
\item[Integrační]{ část se snaží, aby se signál co nejvíce přiblížil námi požadované hodnotě. Toho využijeme v případech, kdy hodnota signálu proporcionální části sama o sobě nedosáhne požadované hodnoty. Integrační regulátor se tedy snaží eliminovat chybu z dlouhodobého hlediska. Uvažujme výstup integračního regulátoru $y_{I}$, konstantu $k$, konstantu značící integrační čas $\tau_{I}$, požadovaný výstup $y_{s}(\tau)$ a aktuální výstup $y(\tau)$, kde $\tau$ nabývá hodnot od 0 až po aktuální čas $t$:
\[ y_{I}(t) = \frac{K_{I}}{\tau_{I}}\int_{0}^{t} (y_{s}(\tau)-y(\tau))d\tau \] } 
\item[Derivační]{ část slouží ke stabilizaci signálu. Snaží se vypočítat velikost chyby v budoucnu, podle toho, jak rychle signál roste v přítomnosti. Derivační regulátor se správně zvolenou konstantou se dokáže rychleji ustálit na požadované hodnotě, což nám může umožnit zvolení agresivnějších konstant a rychlejší celkové odezvy, pokud to systém se kterým pracujeme potřebuje.
\[ y_{D}(t) = k \tau_{D} \frac{d(y_{s}(t)-y(t))}{dt} \]} 
\end{description}

Tyto implikace ve fuzzy logice často popírají některé zákony logiky. Uvažujme pravidlo generalizované modus ponens, které je podobné jako klasické modus ponens s rozdílem, že $A'$ se trochu liší od $A$ a že $B'$ se trochu liší od $B$ \cite{bFoFC}.
\begin{center}
$x$ is $A'$\\
IF $x$ is $A$ THEN $y$ is $B$ \\
\vspace{-6pt}
\rule{140pt}{1pt} \\
$y$ is $B'$.
\end{center}
Fuzzy množinu $B'$ můžeme vyvodit z $A'$ a fuzzy pravidla zapsaného jako fuzzy relace $R_{A\Rightarrow B}$ takto \cite{bFoFC}:
\[\mu_{B'} = \mu_{A'}\circ R_{A\Rightarrow B} \]
Kde operace $\circ$ může značit max-min kompozici fuzzy relací. Nyní mějme fuzzy relaci $R$ pro $High \Rightarrow Low$. Mějme fuzzy množiny: \\ $High = \{(0, 0) , (1000, 0.25) , (2000, 0.5) , (3000, 0.75) , (4000, 1) \}$\\  $Low = \{(0, 1) , (25, 0.75) , (50, 0.5) , (75, 0.25) , (100, 0) \}$\\
Kde $High$ značí nadmořskou výšku v metrech, a $Low$ procenta kyslíku v atmosféře \cite{bFoFC}. Pro operaci implikace použijeme Gödelovu implikaci:
\[ \mu_{A}(x) \Rightarrow \mu_{B}(y) = \left\{
	\begin{array}{ll}
		1  & \mbox{, když } \mu_{A}(x) \leq \mu_{B}(y) \\
		\mu_{B}(y) & \mbox{, když } \mu_{A}(x) > \mu_{B}(y)
	\end{array}
\right. \] 
V tomto konkrétním přkladu bude tedy $A$ = $High$ a $B$ = $Low$.
Pak pravdivostní tabulka $High \Rightarrow Low$ bude vypadat takto \cite{bFoFC}:
\begin{table}
\captionof{table}{Pravdivostní tabulka fuzzy implikace $High \Rightarrow Low$ \cite{bFoFC}}
\centering
$R$ = 
\begin{tabular}{ c | c | c | c | c | c | } 
  & 1 & 0.75 & 0.5 & 0.25 & 0 \\ 
  \hline
  0 & 1 & 1 & 1 & 1 & 1 \\ 
  \hline
  0.25 & 1 & 1 & 1 & 1 & 0 \\ 
  \hline
  0.5 & 1 & 1 & 1 & 0.25 & 0 \\ 
  \hline
  0.75 & 1 & 1 & 0.5 & 0.25 & 0 \\ 
  \hline
  1 & 1 & 0.75 & 0.5 & 0.25 & 0 \\ 
  \hline
\end{tabular}
\end{table}
Čísla na vertikální ose značí $\mu_{High}$ a čísla na horizontální ose značí $\mu_{Low}$. Každý prvek $r_{x\hspace{1pt} y}$ tabulky $R$ je výsledkem implikace $\mu_{High}(x) \Rightarrow \mu_{Low}(y)$. Pokud předpokládáme, že výška je $High$, aplikací modus ponens získáme \cite{bFoFC}:
\[\mu = \mu_{High}\circ R \]
\[\mu = \begin{bmatrix}0 & 0.25 & 0.5 & 0.75 & 1\end{bmatrix} \circ
\begin{bmatrix} 
	1 & 1 & 1 & 1 & 1 \\
	1 & 1 & 1 & 1 & 0 \\
	1 & 1 & 1 & 0.25 & 0 \\
	1 & 1 & 0.5 & 0.25 & 0 \\
	1 & 0.75 & 0.5 & 0.25 & 0 \\
	\end{bmatrix}
\]
\[\mu = \begin{bmatrix}1 & 0.75 & 0.5 & 0.25 & 0\end{bmatrix} = \mu_{Low}\] \\
Pokud je tedy výška $High$ a platí, že $High \Rightarrow Low$ získáme tím $Low$, přesně jak bychom očekávali. Zkusme ale uvažovat výšku $Very \: high$: \[\mu_{Very \: high} = \begin{bmatrix}0 & 0.06 & 0.25 & 0.56 & 1\end{bmatrix}\]
tedy druhou mocninu $\mu_{High}(x)$. Nyní aplikujme modus ponens \cite{bFoFC}: 
\[\mu = \mu_{Very \: High}\circ R \]
\[\mu = \begin{bmatrix}0 & 0.06 & 0.25 & 0.56 & 1\end{bmatrix} \circ
\begin{bmatrix} 
	1 & 1 & 1 & 1 & 1 \\
	1 & 1 & 1 & 1 & 0 \\
	1 & 1 & 1 & 0.25 & 0 \\
	1 & 1 & 0.5 & 0.25 & 0 \\
	1 & 0.75 & 0.5 & 0.25 & 0 \\
	\end{bmatrix}
\]
\[\mu = \begin{bmatrix}1 & 0.75 & 0.5 & 0.25 & 0\end{bmatrix} = \mu_{Low}\]\\
Výsledek není identický s druhou mocninou $\mu_{Low}$. Dokázali jsme tedy, že \cite{bFoFC}: 
\begin{center}
$x$ is $Very \: high$\\
IF $x$ is $High$ THEN $y$ is $Low$ \\
\vspace{-6pt}
\rule{160pt}{1pt} \\
$y$ is $Low$.
\end{center}

\subsection{Tvorba pravidel}
Pravidla vytvoříme pomocí naší intuice. Abychom kyvadlo vyrovnali ve vzpřímené pozici a navrátili vozík na startovní pozici nám postačí 4 pravidla. Mějme tedy čtyři vstupní proměnné: $p\_speed$, $p\_angle$, $c\_pos$, $c\_speed$ a jednu výstupní proměnnou $c\_force$. Aby se kyvadlo vzpřímilo, postačí nám tyto dvě pravidla:
\begin{flushleft}
$R_{1}$: IF $p\_speed$ is $left$ AND $p\_angle$ is $left$ THEN $c\_force$ is $high\_left$

$R_{2}$: IF $p\_speed$ is $right$ AND $p\_angle$ is $right$ THEN $c\_force$ is $high\_right$ 
\end{flushleft}
Prakticky si to můžeme představit tak, že když se kyvadlo naklání na nějakou stranu, posuneme vozík tím stejným směrem a kyvadlo tím vyrovnáme. Dále abychom vozík přesunuli na startovní pozici, použijeme další dvě pravidla:
\begin{flushleft}
$R_{1}$: IF $c\_speed$ is $left$ AND $c\_pos$ is $left$ THEN $c\_force$ is $low\_left$

$R_{2}$: IF $c\_speed$ is $right$ AND $c\_pos$ is $right$ THEN $c\_force$ is $low\_right$ 
\end{flushleft}
Tohle pravidlo se může zdát nesmyslné, protože se může zdát, že tak vozík ještě více oddálíme od požadované startovní pozice. Jenže kdybychom vozík jedoucí od startovní pozice doprava postrčili doleva, vozík bychom ještě více zrychlili špatným směrem. Vozík by se totiž snažil vyrovnat kyvadlo, které je nakloněné směrem od středu. Pokud ho ale trošku postrčíme ve stejném směru, kyvadlo se převrátí a snaha kyvadlo vyrovnat ho povede směrem ke startovní pozici.
\subsection{Volba inferenční metody}
Jako inferenční metodu jsem zvolil základní metodu max-min.

\subsection{Volba defuzzifikační metody}
Jako defuzzifikační metodu jsem zvolil metodu center of gravity, protože vracela v mém případě uspokojivé výsledky.


%% Závěry práce. V jazyce práce a anglicky. Text pro jiný než
%% nastavený jazyk práce (nepovinným parametrem language makra
%% \documentclass, výchozí český) se zadává použitím makra s uvedením
%% jazyka jako nepovinného parametru.
\begin{kiconclusions}
Cílem mé práce bylo popsat funkci fuzzy regulátoru a poskytnout návod, jak takový regulátor dobře navrhnout. Zezačátku jsem popsal řídící systémy a definoval pojem PID regulátor, ke kterému je fuzzy regulátor alternativou. \\ \\
Popsal jsem matematické definice fuzzy logiky a fuzzy množin, na kterých tento regulátor stojí. Později jsem předložil kroky a pravidla, kterých bychom se měli při návrhu fuzzy regulátoru držet. Záměrem práce bylo věnovat se především Mamdaniho fuzzy regulátoru, krátce jsem ale i popsal regulátor typu Takagi-Sugeno-Kang. Popsal jsem nežádoucí situace, které mohou nastat a poskytnul řešení. V práci jsem chtěl zmínit i možnost použití fuzzy implikace ve fuzzy regulátorech, nenalezl jsem ale dostatek informací, abych vyvodil jednoznačný závěr.\\ \\
Na konci práce jsem popsal svůj myšlenkový pochod při sestrojování fuzzy regulátoru pro vozítko s obráceným kyvadlem. K sestrojení jsem použil nadstavby Simulink a Fuzzy logic toolbox programu MATLAB. Ke konstrukci fuzzy regulátoru mi stačila pouze čtyři pravidla a s výsledkem jsem spokojený. \\ \\
Na závěr bych chtěl říct, že mě těší, kolik jsem se toho dozvěděl o fuzzy regulátorech. Myslím si, že zadání bakalářské práce jsem splnil.
\end{kiconclusions}
\newpage


%% Přílohy obsahu textu práce, za makrem \appendix.


%% Obsah přiloženého datového média. Poslední příloha. Upravte podle vlastní
%% práce!
\section{Obsah přiloženého datového média} \label{sec:ObsahMedia}
\begin{description}


\item[\texttt{doc/}] \hfill \\
  Text práce ve formátu PDF, vytvořený s~použitím závazného stylu KI
  PřF UP v~Olomouci pro závěrečné práce, včetně všech příloh,
  a~všechny soubory potřebné pro bezproblémové vygenerování PDF
  dokumentu textu (v~ZIP archivu), tj.~zdrojový text textu, vložené
  obrázky, apod.


\item[\texttt{data/}] \hfill \\
  Soubory pro MATLAB obsahující fuzzy regulátor a systém obráceného kyvadla na vozíku.

\end{description}
\newpage
%% -------------------------------------------------------------------

%% Sazba volitelného seznamu zkratek, za přílohami.
%\printglossary

%% Sazba povinné bibliografie, za přílohami (případně i za seznamem
%% zkratek). Při použití BibLaTeXu použijte makro
%% \printbibliography. jinak prostředí thebibliography. Ne obojí!

%% Sazba i v textu necitovaných zdrojů, při použití
%% BibLaTeXu. Volitelné.
\nocite{*}
%% Vlastní sazba bibliografie při použití BibLaTeXu.
\label{Literatura}
\printbibliography

%% Bibliografie, včetně sazby, při nepoužití BibLaTeXu.
% \begin{thebibliography}{9}
%\bibitem{kniha2} \uppercase{Hawke}, Paul. NanoHttpd: Light-weight HTTP server designed for embedding in other applications. GitHub [online]. 2014-05-12. [cit. 2014-12-06]. Dostupné z: \url{https://github.com/NanoHttpd/nanohttpd}
%
%\bibitem{jeske13} \uppercase{Jeske}, David; \uppercase{Novák}, Josef. Simple HTTP Server in \csharp: Threaded synchronous HTTP Server abstract class, to respond to HTTP requests. CodeProject: For those who code [online]. 2014-05-24. [cit. 2014-12-06]. Dostupné z: \url{http://www.codeproject.com/Articles/137979/Simple-HTTP-Server-in-C}
%
%\bibitem{uzis2012} \uppercase{ÚSTAV ZDRAVOTNICKÝCH INFORMACÍ A STATISTIKY ČR}. Lékaři, zubní lékaři a farmaceuti 2012 [online]. Praha 2, Palackého náměstí 4: Ústav zdravotnických informací a statistiky ČR, 2012 [cit. 2014-12-06]. ISBN 978-80-7472-089-5. Dostupné z: \url{http://www.uzis.cz/publikace/lekari-zubni-lekari-farmaceuti-2012}
% \end{thebibliography}

%% Sazba volitelného rejstříku, za bibliografií.
%\printindex

\end{document}
